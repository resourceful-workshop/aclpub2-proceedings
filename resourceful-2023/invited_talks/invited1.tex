The demand for translation is ever growing and this trend will not stop.
Being able to access the same kind of information is a fundamental prerequisite for equality in society and translation plays a crucial role when fighting discrimination based on language barriers.
Efficient tools and a better coverage of the linguistic diversity in the World are necessary to cope with the amount of material that needs to be handled.
Our mission is to support the development of high quality tools for automatic and computer-assisted translation by providing open services and resources that are independent of commercial interests and profit-driven companies.
Equal information access is a human right and not only a privilege for people who can pay for it.
In this talk I will discuss the current state of OPUS-MT, our project on open neural machine translation and the challenges that we try to tackle with multilingual NLP, transfer learning and data augmentation.
I will report about on-going work on knowledge distillation, the creation of compact models for real-time translation and our work on modularization of neural MT.